\documentclass[a4paper, 11pt]{article}

\usepackage{fullpage}
\usepackage{amsmath}
\usepackage{amsthm}

\usepackage{fontspec}
\usepackage{mathtools}
\usepackage{amsfonts}
\usepackage[usenames,dvipsnames,table]{xcolor}
\usepackage[
  colorlinks  = true,
  linkcolor   = black,
  citecolor   = black,
  pdfauthor   = {Ayberk\ Tosun\ and\ Mart\'{i}n\ H.\ Escard\'{o}},
  pdftitle    = {A\ Construction\ of\ the\ Patch\ Frame\ in\ Univalent\ Type\ Theory},
  pdfsubject  = {Pointfree\ topology},
  pdfkeywords = {},
  bookmarks   = false
]{hyperref}

\usepackage[
  mincrossrefs=999,
  style=numeric,
  backend=biber,
  url=false,
  isbn=false,
  doi=false,
]{biblatex}

\addbibresource{references.bib}

\newtheorem*{ex}{Exercise}
\newtheorem{lem}{Lemma}
\newtheorem{thm}{Theorem}

\theoremstyle{definition}
\newtheorem{prop}{Proposition}
\newtheorem{defn}{Definition}
\newtheorem{example}{Example}

\newcommand{\paren}[1]{\left( #1 \right)}
\newcommand{\setof}[1]{\left\{ #1 \right\}}
\newcommand{\is}{\vcentcolon\equiv}
\DeclareMathOperator{\emptyl}{\epsilon}
\DeclareMathOperator{\cons}{\colon\kern-1.1ex\colon\kern-0.4ex}
\DeclareMathOperator{\append}{\,}

%% Universe notation.
\newcommand{\UU}{\mathcal{U}}
\newcommand{\VV}{\mathcal{V}}
\newcommand{\WW}{\mathcal{W}}

%% Notation for the enumeration function in directification.
\newcommand{\dd}[1]{\mathfrak{d}(#1)}

%% Shorthand for family.
\newcommand{\Fam}[2]{\mathsf{Fam}_{#1}\left(#2\right)}

\newcommand{\define}[1]{\emph{#1}}

\newcommand{\todo}[1]{{\large\color{orange}\textsf{To-do: #1.}}}

\title{A Construction of The Patch Frame in Univalent Type Theory}
\author{Ayberk Tosun \and Mart\'{i}n H.\ Escard\'{o}}

\begin{document}

\maketitle

\begin{abstract}
  \todo{write an abstract}
\end{abstract}

\section{Preliminaries}

\begin{defn}[Family]
  A \define{$\UU$-family on a type} $A$ is a pair $(I, f)$ where $I : \UU$ and $f : I \rightarrow A$. We
  denote the type of $\UU$-families by $\mathsf{Fam}$ i.e.\ $\Fam{\UU}{A} :\equiv \Sigma_{(I : \UU)} I \rightarrow A$.
\end{defn}

\begin{defn}[Directed family]
  A \define{directed family on some type} $A$ is a family $(I, f)$ such that (a) $I$ is (merely)
  inhabited, and (b) for every $i, j : I$, there (merely) exists some $k : I$ such that $f(k)$ is
  the upper bound of $\{ f(i), f(j) \}$.
\end{defn}

Our definition of a frame is parameterised by three universes: one for the carrier set, one for the
order, and one for the index type of families on which the join operation is defined. We adopt the
convention of using universe variables of $\UU$, $\VV$, and $\WW$ for these respectively.

\begin{defn}[Frame]
  A \define{$(\UU, \VV, \WW)$-frame} consists of:
  \begin{itemize}
    \item a set $| F | : \UU$,
    \item a partial order $\_\le\_\ :\ | F | \rightarrow \Omega_\VV$,
    \item a top element $\top : | F |$,
    \item a meet operation i.e.\ an operation $\_\wedge\_\ :\ | F | \rightarrow | F | \rightarrow | F |$ such that $x \wedge y$ is
      the greatest lower bound of $x$ and $y$ for every $x, y : | F |$, and
    \item a join operation i.e.\ an operation $\bigvee\_ : \mathsf{Fam}_{\WW}(| F |) \rightarrow
      | F |$ such that given a family $\{ x_i \}_{i \in I}$, $\bigvee_i x_i$ is the least upper bound
      of $\{ x_i \}_{i : I}$,
  \end{itemize}
  such that binary meets distribute over arbitrary joins:
  \begin{equation*}
    x \wedge \bigvee_{i : I} y_i = \bigvee_{i : I} x \wedge y_i
  \end{equation*}
  for every $x : | F |$, family $\{ y \}_{i : I} : \mathsf{Fam}_{\WW}(| F |)$.
\end{defn}

The nuclei on a frame $F$ correspond to the regular monomorphisms (in the category of locales) on
the locale corresponding to frame $F$.

\begin{defn}[Nucleus]
  A \define{nucleus on a $(\UU, \VV, \WW)$-frame} $F$ is an endofunction $j : | F | \rightarrow | F |$
  satisfying:
  \begin{itemize}
    \item (meet preservation) $j(x \wedge y) = j(x) \wedge j(y)$;
    \item (inflation property) $x \le j(x)$; and
    \item (idempotency) $j(j(x)) \le j(x)$.
  \end{itemize}
  for every $x, y : | F |$.
\end{defn}

If an endofunction satisfies meet preservation and the inflation property but not necessarily
idempotency, it is called a \define{prenucleus}. Prenuclei are closed under composition whereas this
is not generally the case for nuclei.

We write $\mathsf{Nucleus}(F)$ to denote the type of nuclei on a given frame $F$.

\begin{prop}[Prenuclei are monotonic]\label{prop:nuclei-mono}
  Every prenucleus is monotonic.
\end{prop}
\begin{proof}
  Let $j : | F | \rightarrow | F |$ be a nucleus on some frame $F$ and let $x, y : | F |$
  s.t.\ $x \le y$.
  We have:
  \begin{equation*}
    j(x) = j (x \wedge y) = j(x) \wedge j(y) \le j(y),
  \end{equation*}
  using meet preservation and the fact that $x \le y \leftrightarrow x \wedge y = x$.
\end{proof}

\begin{prop}\label{prop:nucl-lemma-1}
  Given any two prenuclei $j$ and $k$ on some frame $F$, $j \circ k$ is an upper bound of $\{ j, k \}$.
\end{prop}
\begin{proof}
  Given any $x : | F |$, we have that $k(x) \le j(k(x))$ by the inflation property of $j$ and that
  $j(x) \le j(k(x))$ by the monotonicity of $j$, since $x \le k(x)$ by the inflation property of $k$.
\end{proof}

\begin{prop}
  Given any nucleus $F$, the type $\mathsf{Nucleus}(F)$ is an set.
\end{prop}

\begin{defn}[Scott-continuity]
  A function $f : | F | \rightarrow | G | $ from a $(\UU, \VV, \WW)$-frame $F$ to a $(\UU, \VV, \WW)$-frame
  $G$ is called \define{Scott-continuous} iff given any \emph{directed}
  $\{ x_i \}_{i : I} : \Fam{\WW}{| F |}$,
  \begin{equation*}
    f\left(\bigvee_{i : I} x_i\right) = \bigvee_{i : I} f(x_i).
  \end{equation*}
\end{defn}

\section{Meet-semilattice of nuclei}

Nuclei form a meet-semilattice when ordered pointwise.

\begin{defn}\label{defn:nuclei-semilattice}
  The type of nuclei on a given frame $F$ forms a meet-semilattice as follows:
  \begin{itemize}
    \item order: given nuclei $j$ and $k$, $j \le k :\equiv \forall x \in | F |.\ j(x) \le k(x)$;
    \item top nucleus: $\_ \mapsto \top_F$; and
    \item meet of two nucleus: $j \wedge k :\equiv x \mapsto j(x) \wedge_F k(x)$.
  \end{itemize}
  The fact that $j \wedge k$ satisfies the nucleus laws is given in Proposition~\ref{prop:nuclei-meet}.
\end{defn}

We denote this $\mathcal{N}(F)$.

\begin{prop}\label{prop:nuclei-meet}
  Given two nuclei $j$ and $k$ on some frame $F$, the function $x \mapsto j(x) \wedge k(x)$ is a nucleus.
\end{prop}
\begin{proof}
  The inflation property can be seen to be satisfied from the inflation properties of $j$ and $k$
  and the fact that $j(x) \wedge k(x)$ is the greatest lower bound of $j(x)$ and $k(x)$. To see that meet
  preservation holds, let $x, y : | F |$. We have:
  \begin{align*}
    j (x \wedge y) \wedge k (x \wedge y) &\quad=\quad j(x) \wedge j(y) \wedge k(x) \wedge k(y) \\
                          &\quad=\quad (j(x) \wedge k(x)) \wedge (j(y) \wedge k(y)).
  \end{align*}
  For idempotency, let $x : | F |$. We have:
  \begin{align*}
    j (j(x) \wedge k(x)) \wedge k(j(x) \wedge k(x)) &\quad\le\quad j(j(x)) \wedge j(k(x)) \wedge k(j(x)) \wedge k(k(x)) \\
                                     &\quad\le\quad j(j(x)) \wedge k(k(x)) & \\
                                     &\quad\le\quad j(x) \wedge k(x).
  \end{align*}
\end{proof}

\begin{prop}\label{prop:sc-nuclei-semilattice}
  The type of Scott-continuous nuclei on a given $(\UU, \VV, \WW)$-frame $F$ forms a
  meet-semilattice in the same way as in Definition~\ref{defn:nuclei-semilattice}. Its top element
  is $\top_{\mathcal{N}(F)}$ which is trivially Scott-continuous. It remains only to be shown that the
  meet of two Scott-continuous nuclei is a Scott-continuous nucleus. Consider two Scott-continuous
  nuclei $j$ and $k$ on $F$ and a directed family $\{ x_i \}_{i : I} : \mathsf{Fam}_{\WW}(| F |)$.
  This follows as:
  \begin{align*}
    (j \wedge k) \left( \bigvee_{i : I} x_i \right)
    &\equiv j \left( \bigvee_{i : I} x_i \right) \wedge k \left( \bigvee_{i : I} x_i \right) & \\
    &= \paren{ \bigvee_{i : I} j(x_i) } \wedge \paren{ \bigvee_{i : I} k(x_i) } & [\text{Scott-continuity of $j$ and $k$}]\\
    &= \bigvee_{(i, j) : I \times I} j(x_i) \wedge k(x_i) & [\text{distributivity}]\\
    &= \bigvee_{i : I} j(x_i) \wedge k(x_i) & [\text{\dag}]
  \end{align*}
  where, for the \dag\ step, we use antisymmetry. The backwards direction is immediate whereas the
  forwards direction follows, essentially, from the monotonicity of nuclei
  (Proposition~\ref{prop:nuclei-mono}). We omit the details.
\end{prop}

\section{Joins}

The nontrivial part of the patch frame construction is the join of a family $\{ k_i \}_{i : I}$ of
Scott-continuous nuclei as defining it pointwise,
\begin{equation*}
  \bigvee_{i : I} k_i \quad:\equiv\quad x \mapsto \bigvee_{i : I} k_i(x),
\end{equation*}
does not work. The problem is that this is not idempotent and hence not a nucleus. Our construction
(that follows Escard\'{o}~\cite{properly-injective}) is based on the idea that, if we start with a
family $\{ k_i \}_{i : I}$ of nuclei, we can consider the family
\begin{equation*}
  \left\{ k_{i_0} \circ \cdots \circ k_{i_n} \right\}_{(i_0, \cdots, i_n) : \mathsf{List}(I)},
\end{equation*}
which will always be directed. However, in a predicative setting, this works only on
Scott-continuous nuclei. Let us write this a bit more precisely as follows:

Before we proceed with the construction, let us note that we use the following notation for lists:
the empty list is denoted $\emptyl$, prepending an element $x$ onto some list $s$ is denoted $x
\cons s$, and the appending of $s$ to the end of list $t$ is denoted $s \append t$.

\begin{defn}
  Given a family $K :\equiv \{ k_i \}_{i \in I}$ of nuclei on a given $(\UU, \VV, \WW)$-frame $F$, we
  define \define{$K^*$} as the family $(\mathsf{List}(I), \mathfrak{d})$ where $\mathfrak{d}$ is
  defined as:
  \begin{align*}
    \dd{\emptyl}   \quad&:\equiv\quad \mathsf{id};  \\
    \dd{i \cons s} \quad&:\equiv\quad \dd{s} \circ k_i.
  \end{align*}
\end{defn}

\begin{prop}\label{prop:app-lemma}
  Let $K :\equiv \{ k_i \}_{i : I}$ be a family of prenuclei on a frame $F$. Given
  any $s, t : \mathsf{List}(I)$,
  \(\dd{s \append t} = \dd{s} \circ \dd{t}.\)
\end{prop}
\begin{proof}
  Straightforward induction, using function extensionality.
\end{proof}

\begin{prop}\label{prop:star-prenucleus}
  Given a family $K :\equiv \{ k_i \}_{i : I}$ of nuclei on a $(\UU, \VV, \WW)$-frame $F$, every $\alpha \in
  K^*$ is a prenucleus, that is, for every $s : \mathsf{List}(I)$, the function $\dd{s}$
  is a prenucleus
\end{prop}
\begin{proof}
  If $s = \emptyl$, we are done as it is immediate that the identity function $\mathsf{id}$ is a
  prenucleus. If $s = i \cons s'$, we need to show that $\dd{s'} \circ k_i$ is a prenucleus. For meet
  preservation, consider some $x, y : | F |$. We have that:
  \begin{align*}
    (\dd{s'} \circ k_i)(x \wedge y)
      &\quad\equiv\quad \dd{s'}(k_i(x \wedge y))                                                       \\
      &\quad=\quad \dd{s'}(k_i(x) \wedge k_i(y))                  & [\text{$k_i$ is a nucleus}]   \\
      &\quad=\quad \dd{s'}(k_i(x)) \wedge \dd{s'}(k_i(y))        & [\text{inductive hypothesis}] \\
      &\quad\equiv\quad (\dd{s'} \circ k_i)(x) \wedge (\dd{s'} \circ k_i)(y).
  \end{align*}
  For the inflation property, consider some $x : | F |$. We have that
  \begin{equation*}
    x \le k_i(x) \le \dd{s'}(k_i(x)),
  \end{equation*}
  by the inflation property of $k_i$ and the inductive hypothesis.
\end{proof}

\begin{prop}\label{prop:star-ub}
  Given a nucleus $j$ and a family $K \is \{ k_i \}_{i : I}$ of nuclei on some frame $F$, if $j$ is
  an upper bound of $K$ then $j$ is an upper bound of $K^*$.
\end{prop}
\begin{proof}
  Let $s : \mathsf{List}(I)$. If $s = \emptyl$, we have that $\mathsf{id}(x) \equiv x \le j(x)$.
  If $s = i \cons s'$, then we have
  \begin{align*}
       K^*_{s'}(K_i(x))
  \quad&\le\quad K^*_{s'}(j(x)) & [\text{monotonicity of $K^*_{s'}$ (Prop.~\ref{prop:star-prenucleus} and Prop.~\ref{prop:nuclei-mono})}] \\
  \quad&\le\quad j(j(x))        & [\text{inductive hypothesis}] \\
  \quad&\le\quad j(x).          & [\text{idempotency of $j$}]
  \end{align*}
\end{proof}

\begin{prop}
  Given a family $\{ k_i \}_{i : I}$ of Scott-continuous nuclei, every
  prenucleus $\alpha \in K^*$, is Scott-continuous
\end{prop}
\begin{proof}
  In the base case of $s = \emptyl$, it is trivial that the identity
  prenucleus is Scott-continuous. For the case of $s = i \cons s'$, let
  $\{ x_i \}_{i : I}$ be a directed family on $F$. The result follows equationally
  as follows:
  \begin{align*}
    (\dd{s'} \circ k_i) \left(\bigvee_i x_i\right)
    &\quad\equiv\quad \dd{s'}\left(k_i \left( \bigvee_i x_i \right)\right) & \\
    &\quad=\quad \dd{s'}\left(\bigvee_{i : I} \left( k_i(x_i) \right)\right) & [\text{Scott-continuity of $k_i$}]\\
    &\quad=\quad \bigvee_{i : I} \dd{s'}(k_i(x_i)) & [\text{inductive hypothesis}].
  \end{align*}
  Note that to be able to appeal to the inductive hypothesis, it must be shown that
  $\{ k_i(x_i) \}_{i : I}$ is a directed family which follows from \ldots \todo{complete}
\end{proof}

\begin{prop}
  Given a family $K :\equiv \{ k_i \}_{i : I}$ of nuclei on some frame $F$, the family $K^*$ is directed.
\end{prop}
\begin{proof}
  $K^*$ is always inhabited by $\dd{\emptyl}$. Upwards-closure also holds since, given $s, t :
  \mathsf{List}(I)$, $\dd{s \append t}$ is the upper bound of $\{ \dd{s}, \dd{t} \}$:
  $\dd{s \append t} = \dd{s} \circ \dd{t}$ (by Proposition~\ref{prop:app-lemma}) which is the upper
  bound of $\{ \dd{s}, \dd{t} \}$ by Proposition~\ref{prop:nucl-lemma-1}.
\end{proof}

To be able to define the patch frame on $F$, we will also need some lemmas about nuclei in general.

\begin{prop}
  Given a nucleus $j$ on some frame $F$ and a family $K \is \{ k_i \}_{i : I}$ of nuclei on $F$, if
  we denote by $L$ the set $\{ j \wedge k ~|~ k \in K \}$, we have that $L^*_{s}$ is a lower bound of
  $\{ K^*_{s}, j \}$.
\end{prop}
\begin{proof}
  Let $s : \mathsf{List}(I)$ and $x : | F |$. In the base case of $s \equiv \emptyl$, we are done
  as both sides of the inequality reduce to the application of $\mathsf{id}(x)$. If we have
  $s \equiv i \append s'$, we have that
  \begin{align*}
    L^*_{i \cons s'}(x)
      \quad\equiv\quad L^*_{s'}(L_i(x))
     &\quad\equiv\quad L^*_{s'}(j(x) \wedge k_i(x)) \\
     &\quad\le\quad K^*_{s'}(j(x) \wedge k_i(x)) & [\text{inductive hypothesis}] \\
     &\quad\le\quad K^*_{s'}(j(x)) \wedge K^*_{s'}(k_i(x)) & [K^*_{s'}\ \text{is a prenucleus}] \\
     &\quad\le\quad K^*_{s'}(k_i(x)),
  \end{align*}
  and that
  \begin{align*}
    L^*_{i \cons s'}(x)
    \quad\equiv\quad L^*_{s'}(L_i(x)) 
    &\quad\equiv\quad L^*_{s'}(j(x) \wedge k_i(x)) & \\
    &\quad\le\quad j(j(x) \wedge k_i(x)) & [\text{inductive hypothesis}] \\
    &\quad\le\quad j(j(x)) \wedge j(k_i(x)) & [\text{$j$ preserves meets}]\\
    &\quad\le\quad j(j(x)) & \\
    &\quad=\quad j(x), & [\text{$j$ is idempotent}]
  \end{align*}
  which is to say $L^*_{i \cons s'}(x)$ is lower than both of $K^*_{i \cons s'}(x)$ and $j(x)$.
\end{proof}

\begin{defn}[Join of a family of Scott-continuous nuclei]
  Let $K \is \{ k_i \}_{i : I}$ be a family of Scott-continuous nuclei. Its join is given by
  the operation $\bigvee^N$ defined as:
  \begin{equation*}
    \bigvee^N_{i} k_i \quad\is\quad x \mapsto \bigvee \left\{ \alpha(x) ~|~ \alpha \in K^* \right\}.
  \end{equation*}
  We need to show that $\bigvee^N_{i} k_i$ is
    (a) a nucleus,
    (b) Scott-continuous, and
    (c) the least upper bound of $K$.

  \paragraph{(a).} The inflation property is direct. For meet preservation, consider some
  $x, y : | F |$. We have:
  \begin{align*}
         \left(\bigvee^N_{i} k_i\right)(x)
    &\quad\equiv\quad \bigvee_{\alpha \in K^*} \alpha(x \wedge y)                                                                   \\
    &\quad=\quad \bigvee_{\alpha \in K^*} \alpha(x) \wedge \alpha(y)                                & [\text{$\alpha$ is a prenucleus}]  \\
    &\quad=\quad \bigvee_{\beta, \gamma \in K^*} \beta(x) \wedge \gamma(y)                             & [\text{\todo{complete}}]      \\
    &\quad=\quad \paren{\bigvee_{\beta \in K^*} \beta(x)} \wedge \paren{\bigvee_{\gamma \in K^*} \gamma(y)}    & [\text{distributivity}]       \\
    &\quad\equiv\quad \paren{\bigvee^N_{i : I} k_i}(x) \wedge \paren{\bigvee^N_{i : I} k_i}(y).
  \end{align*}
  For idempotency, let $x : | F |$. We have that:
  \begin{align*}
    \left(\bigvee^N_{i} k_i\right)\left(\left(\bigvee^N_{i} k_i\right)(x)\right)
    &\quad\equiv\quad \bigvee \setof{ \bigvee \setof{ \alpha(\beta(x)) ~|~ \beta in K^* } ~|~ \alpha \in K^* } \\
    &\quad\le\quad \bigvee_{\alpha, \beta \in K^*} \alpha(\beta(x)) & [\text{\todo{explain}}]\\
    &\quad\le\quad \bigvee_{\alpha \in K^*} \alpha(x) & [\text{\todo{explain}}]\\
    &\quad\equiv\quad \paren{\bigvee^N_i k_i}(x).
  \end{align*}

  \paragraph{(b).} Let $U \is \{ x_j \}_{j : J}$ be a directed family over $| F |$. Then
  \begin{align*}
       \bigvee \setof{ \alpha\left(\bigvee_j x_j\right) ~|~ \alpha \in K^* }
  &\quad=\quad \bigvee \setof{ \bigvee \setof{ \alpha(x_j) ~|~ x_j \in U } ~|~ \alpha \in K^* }  & [\text{Scott-continuity of $\alpha$}] \\
  &\quad=\quad \bigvee \setof{ \bigvee \setof{ \alpha(x_j) ~|~ \alpha : K^* } ~|~ x_j \in U }, & [\text{joins commute}]           \\
  \end{align*}
  as required.

  \paragraph{(c).} The fact that $\bigvee^N_i k_i$ is an upper bound of $K$ is easy to verify:
  given some $k_i$ and $x : | F |$, $k_i(x) \in \{ \alpha(x) ~|~ \alpha \in K^* \}$ since $k_i \in K^*$. To see that
  it is \emph{the least} upper bound, consider a Scott-continuous nucleus $j$ that is an upper bound
  of $K$. Let $x : | F |$. We need to show that $\left(\bigvee^N_i k_i\right)(x) \le j(x)$. We know by
  Proposition~\ref{prop:star-ub} that $j$ is an upper bound of $K^*$, since it is an upper bound of
  $K$, which is to say $K^*_{s}(x) \le j(x)$ for every $s : \mathsf{List}(I)$ i.e.\ $j(x)$ is an
  upper bound of the family $\setof{ \alpha(x) ~|~ \alpha \in K^* }$. Since $\left(\bigvee^N_i k_i\right)(x)$ is the
  least upper bound of this family by definition, it follows that it is below $j(x)$.
\end{defn}

\begin{prop}[Distributivity]
  Given a nucleus $j$ and a family of nuclei $K \is \{ k_i \}_{i : I}$ on some frame $F$, we have
  that $j \wedge \left(\bigvee_{i : I} k_i\right) = \bigvee_{i : I} j \wedge k_i$.
\end{prop}
\begin{proof}
  \todo{complete the proof}
\end{proof}

\printbibliography

\end{document}
