\documentclass[11pt,a4paper]{article}


%%%%%%%%%%%%%%%%%%%%%%%%%%%%%%%
%%%% PACKAGES
%%%%%%%%


\usepackage[margin=3cm]{geometry}
\usepackage{amsmath}
\usepackage{amsthm}

\usepackage{fontspec}
\usepackage{mathtools}
\usepackage{amsfonts}
\usepackage[usenames,dvipsnames,table]{xcolor}

\usepackage[
  mincrossrefs=999,
  style=numeric,
  backend=biber,
  url=false,
  isbn=false,
  doi=false,
]{biblatex}

\addbibresource{references.bib}

\newtheorem*{ex}{Exercise}
\newtheorem{lem}{Lemma}
\newtheorem{thm}{Theorem}

\theoremstyle{definition}
\newtheorem{prop}{Proposition}
\newtheorem{defn}{Definition}
\newtheorem{example}{Example}

\newcommand{\is}{:\equiv}

\newcommand{\framenm}{\mathsf{Frame}}

\newcommand{\isnuclearnm}{\mathsf{isNuclear}}
\newcommand{\isnuclear}[1]{\isnuclearnm{}\left(#1\right)}
\newcommand{\nucleus}{\mathsf{Nucleus}}
\newcommand{\abs}[1]{\left| #1 \right|}
\newcommand{\hprop}{\Omega}

\newcommand{\meet}[2]{#1 \wedge #2}

\newcommand{\pow}[1]{\mathcal{P}\left(#1\right)}
\newcommand{\pity}[3]{\prod_{(#1~:~#2)} #3}
\newcommand{\sigmaty}[3]{\sum_{(#1~:~#2)} #3}
\newcommand{\univ}{\mathcal{U}}

\title{A Construction of The Patch Frame in Univalent Type Theory}
\author{Ayberk Tosun \and Mart\'{i}n Escard\'{o}}

\begin{document}

\maketitle

\begin{abstract}
\end{abstract}

\section{Preliminaries}

\begin{defn}[Nucleus]\label{defn:nucleus}
  Let $F : \framenm{}$ be a frame and $j : \abs{F} \rightarrow \abs{F}$, an endofunction
  on it. We say that $j$ is \emph{nuclear} iff the following condition holds:
  \begin{alignat*}{5}
    \isnuclearnm{}\quad&:\quad &&(\abs{F} \rightarrow \abs{F}) \rightarrow \hprop{} && && &&              \\
    \isnuclear{j} \quad&\is\quad
       &&\pity{x~y}{\abs{F}}{j(\meet{x}{y}) &&~=~ &&\meet{j(x)}{j(y)}}   &&\qquad[\text{meet-preservation}]\\
      &\hspace{0.5em}\times\quad &&\pity{x~~}{\abs{F}}{x &&~\sqsubseteq~ &&j(x)}            &&\qquad[\text{inflation}]  \\
      &\hspace{0.5em}\times\quad &&\pity{x~~}{\abs{F}}{j(j(x)) &&~\sqsubseteq~ &&j(x)}      &&\qquad[\text{idempotence}].
  \end{alignat*}
  Propositionality follows from the closure of $\Omega$ under $\prod$ and $\sum$ types along
  with the fact that the carrier type is an h-set. The type of nuclei is then
  just the $\sum$ type collecting all nuclear endofunctions on a frame:
  \begin{equation*}
    \nucleus{}(F) \quad\is\quad \sigmaty{j}{\abs{F} \rightarrow \abs{F}}{\isnuclear{j}}.
  \end{equation*}
\end{defn}
\printbibliography

\end{document}
