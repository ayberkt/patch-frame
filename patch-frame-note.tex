\documentclass[a4paper, 11pt]{article}


%%%%%%%%%%%%%%%%%%%%%%%%%%%%%%%
%%%% PACKAGES
%%%%%%%%


\usepackage{fullpage}
\usepackage{amsmath}
\usepackage{amsthm}

\usepackage{fontspec}
\usepackage{mathtools}
\usepackage{amsfonts}
\usepackage[usenames,dvipsnames,table]{xcolor}

\usepackage[
  mincrossrefs=999,
  style=numeric,
  backend=biber,
  url=false,
  isbn=false,
  doi=false,
]{biblatex}

\addbibresource{references.bib}

\newtheorem*{ex}{Exercise}
\newtheorem{lem}{Lemma}
\newtheorem{thm}{Theorem}

\theoremstyle{definition}
\newtheorem{prop}{Proposition}
\newtheorem{defn}{Definition}
\newtheorem{example}{Example}

\newcommand{\paren}[1]{\left( #1 \right)}
\DeclareMathOperator{\cons}{\mathtt{::}}

%% Universe notation.
\newcommand{\UU}{\mathcal{U}}
\newcommand{\VV}{\mathcal{V}}
\newcommand{\WW}{\mathcal{W}}

\title{A Construction of The Patch Frame in Univalent Type Theory}
\author{Ayberk Tosun \and Mart\'{i}n Escard\'{o}}

\begin{document}

\maketitle

\begin{abstract}
  Lorem ipsum.
\end{abstract}

\section{Preliminaries}

\begin{defn}[Family]
  A ``$\UU$-family on type $A$'' is\footnote{%
    It would be reasonable to require $I$ to be an h-set and $f$ to be injective but we are
    providing a direct translation of the definition in our formalisation. This definition might
    have to be updated with these requirements.
  }
  simply a pair $(I, f)$ where $I : \UU$ and $f : I \rightarrow A$. We denote the type of
  $\UU$-families by $\mathsf{Fam}$ i.e.\ $\mathsf{Fam}_{\UU}(A) :\equiv \Sigma_{(I : \UU)} I \rightarrow A$.
\end{defn}

\begin{defn}[Directed family]
  A family $(I, f)$ on some type $A$ is called \emph{directed} iff (1) $I$ is inhabited, and (2) for
  every $i, j : I$, there exists some $k : I$ such that $f(k)$ is the upper bound of
  $\{ f(i), f(j) \}$.
\end{defn}

Our definition of a frame is parameterised by three universes: one for the carrier set, one for the
order, and one for the index type of families on which the join operation is defined. We adopt the
convention of using universe variables of $\UU$, $\VV$, and $\WW$ for these respectively.

\begin{defn}[Frame]
  A $(\UU, \VV, \WW)$-frame consists of:
  \begin{itemize}
    \item an h-set $| F | : \UU$,
    \item a partial order $\_\le\_\ :\ | F | \rightarrow \Omega_\VV$,
    \item a top element $\top : | F |$,
    \item a meet operation i.e.\ an operation $\_\wedge\_\ :\ | F | \rightarrow | F | \rightarrow | F |$
      such that $x \wedge y$ is the GLB of $x$ and $y$ for every $x, y : | F |$, and
    \item a join operation i.e.\ an operation $\bigvee\_ : \mathsf{Fam}_{\WW}(| F |) \rightarrow
      | F |$ such that given a family $\{ x_i \}_{i \in I}$, $\bigvee_i x_i$ is the LUB
      of $\{ x_i \}_{i \in I}$,
  \end{itemize}
  such that binary meets distribute over arbitrary joins:
  \begin{equation*}
    x \wedge \bigvee_{i : I} y_i = \bigvee_{i : I} x \wedge y_i
  \end{equation*}
  for every $x : | F |$, family $\{ y \} : \mathsf{Fam}_{\WW}(| F |)$.
\end{defn}

A nucleus on a frame $F$ is an equivalent characterisation of a regular monomorphism on the locale
corresponding to $F$.

\begin{defn}[Nucleus]
  A nucleus on a $(\UU, \VV, \WW)$-frame $F$ is an endofunction
  $j : | F | \rightarrow | F |$ satisfying:
  \begin{itemize}
    \item (meet-preservation) $j(x \wedge y) = j(x) \wedge j(y)$;
    \item (inflation) $x \le j(x)$; and
    \item (idempotence) $j(j(x)) \le j(x)$.
  \end{itemize}
  for every $x, y : | F |$.
\end{defn}

If an endofunction satisfies meet-preservation and inflation but not idempotence, it is called a
\emph{prenucleus}.

We write $\mathsf{Nucleus}(F)$ to denote the type of nuclei on a given frame $F$.

\begin{prop}[Nuclei are monotonic]\label{prop:nuclei-mono}
  Every nucleus is monotonic.
\end{prop}
\begin{proof}
  Let $j : | F | \rightarrow | F |$ be a nucleus on some frame $F$ and let $x, y : | F |$ s.t.\ $x \le y$.
  We have:
  \begin{equation*}
    j(x) = x \wedge j(x) \le x \le y \le j(y).
  \end{equation*}
\end{proof}

\begin{prop}
  Given any nucleus $F$, $\mathsf{Nucleus}(F)$ is an h-set.
\end{prop}

\begin{defn}[Scott-continuity]
  A function $f : | F | \rightarrow | G | $ from a $(\UU, \VV, \WW)$-frame $F$ to a $(\UU, \VV, \WW)$-frame
  $G$ is called \emph{Scott-continuous} iff given any \emph{directed}
  $\{ x_i \}~:~\mathsf{Fam}_{\WW}(| F |)$,
  \begin{equation*}
    f\left(\bigvee^F_{i : I} x_i\right) = \bigvee^G_{i : I} f(x_i).
  \end{equation*}
\end{defn}

\section{Meet-semilattice of nuclei}

Nuclei form a meet-semilattice when ordered pointwise.

\begin{defn}\label{defn:nuclei-semilattice}
  The type of nuclei on a given frame $F$ forms a meet-semilattice as follows:
  \begin{itemize}
    \item order: given nuclei $j$ and $k$, $j \le k :\equiv \forall x \in | F |.\ j(x) \le k(x)$;
    \item top nucleus: $\_ \mapsto \top_F$; and
    \item meet of two nucleus: $j \wedge k :\equiv x \mapsto j(x) \wedge_F k(x)$.
  \end{itemize}
  The fact that $j \wedge k$ satisfies the nucleus laws is given in Proposition~\ref{prop:nuclei-meet}.
\end{defn}

We denote this $\mathcal{N}(F)$.

\begin{prop}\label{prop:nuclei-meet}
  Given two nuclei $j$ and $k$ on some frame $F$, the function $x \mapsto j(x) \wedge k(x)$ is a nucleus.
\end{prop}
\begin{proof}
  Inflation is direct from the inflation of $j$ and $k$ and the fact that $j(x) \wedge k(x)$ is the GLB
  of $j(x)$ and $k(x)$. To see that meet-preservation holds, let $x, y : | F |$. We have:
  \begin{align*}
    j (x \wedge y) \wedge k (x \wedge y) &\quad=\quad j(x) \wedge j(y) \wedge k(x) \wedge k(y) \\
                          &\quad=\quad (j(x) \wedge k(x)) \wedge (j(y) \wedge k(y)).
  \end{align*}
  For idempotence, let $x : | F |$. We have:
  \begin{align*}
    j (j(x) \wedge k(x)) \wedge k(j(x) \wedge k(x)) &\quad\le\quad j(j(x)) \wedge j(k(x)) \wedge k(j(x)) \wedge k(k(x)) \\
                                     &\quad\le\quad j(j(x)) \wedge k(k(x)) & \\
                                     &\quad\le\quad j(x) \wedge k(x).
  \end{align*}
\end{proof}

\begin{defn}\label{defn:sc-nuclei-semilattice}
  The type of Scott-continuous nuclei on a given $(\UU, \VV, \WW)$-frame $F$ forms a
  meet-semilattice in the same way as in Definition~\ref{defn:nuclei-semilattice}. Its top element
  is $\top_{\mathcal{N}(F)}$ which is trivially Scott-continuous. It remains only to be shown that the
  meet of two Scott-continuous nuclei is a Scott-continuous nucleus. Consider two Scott-continuous
  nuclei $j$ and $k$ on $F$ and a directed family $\{ x_i \}_{i : I} : \mathsf{Fam}_{\WW}(| F |)$.
  This follows as:
  \begin{align*}
    (j \wedge k) \left( \bigvee_{i : I} x_i \right)
    &\equiv j \left( \bigvee_{i : I} x_i \right) \wedge k \left( \bigvee_{i : I} x_i \right) & \\
    &= \paren{ \bigvee_{i : I} j(x_i) } \wedge \paren{ \bigvee_{i : I} k(x_i) } & [\text{Scott-continuity of $j$ and $k$}]\\
    &= \bigvee_{(i, j) : I \times I} j(x_i) \wedge k(x_i) & [\text{distributivity}]\\
    &= \bigvee_{i : I} j(x_i) \wedge k(x_i) & [\text{\dag}]
  \end{align*}
  where, for the \dag\ step, we use antisymmetry. The backwards direction is immediate whereas the
  forwards direction follows, essentially, from the monotonicity of nuclei
  (Proposition~\ref{prop:nuclei-mono}). We omit the details.
\end{defn}

\section{Joins}

The nontrivial part of the patch frame construction is the join of a family $\{ k_i \}_{i : I}$ of
Scott-continuous nuclei as defining it pointwise,
\begin{equation*}
  \bigvee_{i : I} k_i \quad:\equiv\quad x \mapsto \bigvee_{i : I} k_i(x),
\end{equation*}
does not work. The problem is that this is not idempotent and hence not a nucleus. Our construction
(that follows Escard\'{o}~\cite{properly-injective}) is based on the idea that, if we start with a
family $\{ k_i \}_{i : I}$ of nuclei, we can consider the family
\begin{equation*}
  \left\{ k_{i_0} \circ \cdots \circ k_{i_n} \right\}_{(i_0, \cdots, i_n) : \mathsf{List}(I)},
\end{equation*}
which will always be directed. However, in a predicative setting, this works only on
Scott-continuous nuclei. Let us write this a bit more precisely as follows:

\begin{defn}
  Let $K :\equiv \{ k_i \}_{i \in I}$ be a family of nuclei on a given
  $(\UU, \VV, \WW)$-frame $F$. We define $K^*$ as
  the family $(\mathsf{List}(I), \mathfrak{d})$ where $\mathfrak{d}$ is defined
  as:
  \begin{align*}
    \mathfrak{d}(\mathtt{[\ ]})      \quad&:\equiv\quad \mathsf{id}; \\
    \mathfrak{d}(i \cons is) \quad&:\equiv\quad \mathfrak{d}(is) \circ k_i.
  \end{align*}
\end{defn}

We use the following proposition to construct the patch frame.

\begin{prop}
  Given a family $K$ of Scott-continuous nuclei, $K^*$ is always a directed
  family of Scott-continuous prenuclei (i.e.\ a prenucleus is an endofunction
  satisfying all nuclei laws except idempotence).
\end{prop}

\begin{defn}[Patch frame over a frame $F$]
  Let $F$ be a $(\UU, \VV, \WW)$-frame. We define the patch frame on $F$ as
  follows:
  \begin{itemize}
    \item the underlying meet-semilattice is as given in
      Defn.~\ref{defn:nuclei-semilattice};
    \item given a family $K :\equiv \{ k_i \}_{i \in I}$ of Scott-continuous nuclei,
      its join is defined as:
      \begin{equation*}
        \bigvee K \quad:\equiv\quad x \mapsto \{ \alpha(x) ~|~ \alpha \in K^* \}.
      \end{equation*}
  \end{itemize}
  The proof of the distributivity law is not straightforward and we omit it.
\end{defn}

\printbibliography

\end{document}
